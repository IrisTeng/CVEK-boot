\documentclass[11pt]{article}

% change document font family to Palatino, and code font to Courier
\usepackage{mathpazo} % add possibly `sc` and `osf` options
\usepackage{eulervm}
\usepackage{courier}
%allow formula formatting
\usepackage{amsmath}
\DeclareMathAlphabet\mathbfcal{OMS}{cmsy}{b}{n}
\usepackage{comment}
\usepackage{enumerate}
\usepackage{xcolor}
\usepackage{algorithm}
\usepackage{algpseudocode}
%title positon
\usepackage{titling} %fix title
\setlength{\droptitle}{-6em}   % Move up the title 
%change section title font size
\usepackage{titlesec} 
\titleformat{\section}
  {\normalfont\fontsize{12}{15}}{\thesection}{1em}{}
\titleformat{\subsection}
  {\normalfont\fontsize{12}{13}}{\thesection}{1em}{}
\usepackage{verbatim}
% change page margin
\usepackage[margin=1 in]{geometry} 
\usepackage{subfigure}
%allow inserting multiple graphs
\usepackage{subfig}
\usepackage{graphicx}
\usepackage{float}
%allow code chunks
\usepackage{listings}
\lstset{basicstyle=\footnotesize\ttfamily,breaklines=true}
\lstset{frame=lrbt,xleftmargin=\fboxsep, xrightmargin=-\fboxsep}

%declare sum int sign
\DeclareMathOperator*{\SumInt}{%
\mathchoice%
  {\ooalign{$\displaystyle\sum$\cr\hidewidth$\displaystyle\int$\hidewidth\cr}}
  {\ooalign{\raisebox{.14\height}{\scalebox{.7}{$\textstyle\sum$}}\cr\hidewidth$\textstyle\int$\hidewidth\cr}}
  {\ooalign{\raisebox{.2\height}{\scalebox{.6}{$\scriptstyle\sum$}}\cr$\scriptstyle\int$\cr}}
  {\ooalign{\raisebox{.2\height}{\scalebox{.6}{$\scriptstyle\sum$}}\cr$\scriptstyle\int$\cr}}
}

\usepackage{"./macro/GrandMacros"}
\usepackage{"./macro/Macro_BIO235"}
\begin{document}
\setlength{\parindent}{0pt}
%\setcounter{equation}{0}
%\renewcommand{\theequation}{1.1.\arabic{equation}}
%\paragraph{AIC and small sample correction}\mbox{}\\
%%%%%%%%%%%%%%%%%%%%%%%%%%%%%%%%%%%%%%%%%%%%%
%%%%%%%%%%%% TItle page with contents %%%%%%%%%%%%%%%
%%%%%%%%%%%%%%%%%%%%%%%%%%%%%%%%%%%%%%%%%%%%%

\title{CVEK-boot Report\\Wenying Deng \vspace{-1ex}}

\pretitle{\begin{flushright}\normalsize}
\posttitle{\par\end{flushright}}
\author{}
\date{}
\vspace{-10em}
\maketitle
\vspace{-8em}

%%%%%%%%%%%%%%%%%%%%%%%%%%%%%%%%%%%%%%%%%%%%%
%%%%%%%%%%%% Formal Sections %%%%% %%%%%%%%%%%%%%%
%%%%%%%%%%%%%%%%%%%%%%%%%%%%%%%%%%%%%%%%%%%%%
\vspace{2em}
\section*{{\bf 1. Introduction}}
\section*{{\bf 2. Method}}
\subsection*{{1. CVEK}}
\setcounter{equation}{0}
\renewcommand{\theequation}{2.1.\arabic{equation}}
\subsubsection*{{1.1. main algorithm, need to specify h before}}
We need a kernel estimation strategy that is \textsl{flexible} so that it does not underfit under the null, yet \textsl{stable} so that it does not overfit under the alternative. To this end, we propose estimating $h$ using the ensemble of a library of fixed base kernels $\{k_d\}_{d=1}^D$:
\begin{align}
\hat{h}(\bx)=\sum_{d=1}^D u_d\hat{h}_d(\bx), \quad \bu \in \Delta=\{\bu | \bu \geq 0, \parallel \bu \parallel_1=1\}
\end{align}
where $\hat{h}_d$ is the kernel predictor generated by $d^{th}$ base kernel $k_d$.\\
To be more specific, for each given basis kernel $\{k_d\}_{d=1}^D$, we first estimate $\hat{\bh}_d=\bK_d(\bK_d+\hat{\lambda}_d\bI)^{-1}\by$, the prediction based on $d^{th}$ kernel, where the tuning parameter $\hat{\lambda}_d$ is selected by minimizing one of the four objective functions given in section 1.2. Denote the estimated error for $d^{th}$ kernel as $\hat{\epsilon}_d$ and $\bA_{d,\lambda}=\bK_d(\bK_d+\lambda\bI)^{-1}$. After that, we introduce two ways to ensemble these $D$ kernel predictors $\{\hat{h}_d\}_{d=1}^D$ in section 1.3. Then we get the ensemble matrix $\hat{\bA}$ and can estimate the ensemble kernel matrix $\hat{\bK}$ by solving:
\begin{align*}
\hat{\bK}(\hat{\bK}+\lambda \bI)^{-1}=\hat{\bA}
\end{align*}
Specifically, if we denote $\bU_A$ and $\{\delta_{A,k}\}_{k=1}^n$ the eigenvector and eigenvalues of $\hat{\bA}$, then $\hat{\bK}$ adopts the form:
\begin{align*}
\hat{\bK}=\bU_A diag(\frac{\delta_{A,k}}{1-\delta_{A,k}})\bU_A^T
\end{align*}
 \vspace{1em}
For example, a complete summary of the proposed procedure using LooCV to choose tuning parameter and cross-validation to ensemble kernel predictors is available in Algorithm 1.

%\subsubsection{Cross Validated Ensemble Kernel}

\begin{algorithm}
\caption{Cross Validated Ensemble Kernel (CVEK)} 
\label{alg:cvke}
\begin{algorithmic}[1]
\Procedure{CVEK}{}\newline
\textbf{Input:} A library of kernels $\{k_d\}_{d=1}^D$, Data $(\by, \bx)$\newline
\textbf{Output:} Ensemble Kernel Matrix $\widehat{\bK}$\newline
$\#$ \texttt{Stage 1: Estimate $\lambda$ and CV error for each kernel}
\For{$d = 1$ to $D$}
\State{$\bK_d = \bK_d/tr(\bK_d)$}
\State{$\widehat{\lambda}_d = \mbox{\textit{argmin}} \; 
\texttt{LOOCV}\Big(\lambda | \bK_d \Big)$} 
\State{$\widehat{\epsilon}_d = 
\texttt{CV}\Big(\widehat{\lambda}_d | \bK_d \Big)$}
\EndFor 
\newline
$\#$ \texttt{Stage 2: Estimate ensemble weights $\bu_{D \times 1} = \{u_1, \dots, u_D\}$}
\State{
$\widehat{\bu} = \underset{\bu \in \Delta}{argmin} \; ||\sum_{d=1}^D u_d \widehat{\epsilon}_d||^2 \qquad 
\mbox{where} \quad
\Delta = \{\bu| \bu \geq 0, ||\bu||_1 = 1 \}$}
\newline
$\#$ \texttt{Stage 3: Assemble the ensemble kernel matrix $\widehat{\bK}_{ens}$}
\State{$\widehat{\bA} = \sum_{d=1}^D 
\widehat{\mu}_d\bA_{\widehat{\lambda}_d, k_d}$}
\State{$\bU_A, \bdelta_{A} = 
\texttt{spectral\_decomp}(\widehat{\bA})$}
\State{$\lambda_{\bK} = min \Big(1,  
(\sum_{k=1}^n \frac{\delta_{A, k}}{1 - \delta_{A,k}})^{-1}, 
min \big(\{\widehat{\lambda}_d\}_{d=1}^D \big)
\Big)$}
\State{$\widehat{\bK} = 
\lambda_{\bK} * 
\widehat{\bU}_A \; 
diag\Big( \frac{\delta_{A,k}}{1 - \delta_{A,k}} \Big) \;
\widehat{\bU}_A^T$}
\EndProcedure
\end{algorithmic}
\end{algorithm}

\subsubsection*{{1.2. tuning parameter selection}}
Models may provide a good fit to the training data, but it will not fit sufficiently well to the test data. Tuning parameter could be chosen to address this problem. Here we define four objective functions in terms of tuning parameter $\lambda \in \Lambda$ to be minimized. Denote
\begin{align}
\bA_\lambda=\bK(\bX, \bX)[\bK(\bX, \bX)+\lambda \bI]^{-1}
\end{align}
In this way, $tr(\bA_\lambda)$ is the effective number of model parameters, excluding $\beta_0$ and $\sigma^2$. It decreases monotonically with $\lambda>0$. 

\paragraph{cross-validation: K fold and loo$^{1}$}\mbox{}\\
Cross validation is probably the simplest and most widely used method for estimating prediction error. Suppose we do a $K$-fold cross-validation, which partitions observations into $K$ groups, $\kappa(1),...,\kappa(K)$, and calculates $\bA_\lambda$ $K$ times, each time leaving out group $\kappa(i)$, to get $\bA_\lambda^{-\kappa(1)}, \bA_\lambda^{-\kappa(2)}$, etc. For $\bA_\lambda^{-\kappa(i)}$, cross-validated residuals are calculated on the observations in $\kappa(i)$, which did not contribute to estimating $\bA$. The objective function estimated prediction error and is the sum of the squared cross-validated residuals:
\begin{align}
\lambda_{K-CV}=\underset{\lambda \in \Lambda}{argmin}\;\Big\{log\sum_{i=1}^K[\by_{\kappa(i)}-\bA_\lambda^{-\kappa(i)}\by_{\kappa(i)}]^T[\by_{\kappa(i)}-\bA_\lambda^{-\kappa(i)}\by_{\kappa(i)}]\Big\}
\end{align}
LooCV is the situation when $K=n$. In this case, we can write our objective function as:
\begin{align}
\lambda_{n-CV}=\underset{\lambda \in \Delta}{argmin}\;\Big\{log\;\by^T(\bI_n-\bA_\lambda)\by-2log[\bI_n-diag(\bA_\lambda)-\frac{1}{n}\bI_n]\Big\}
\end{align}

\begin{align*}
\lambda_{n-CV}=\underset{\lambda \in \Delta}{argmin}\;\Big\{log\;\by^T(\bI_n-\bA_\lambda)\by-log[(diag(\frac{n-1}{n}\bI_n-\bA_\lambda))^T(diag(\frac{n-1}{n}\bI_n-\bA_\lambda))]\Big\}
\end{align*}

\begin{align*}
\lambda_{n-CV}=\underset{\lambda \in \Delta}{argmin}\;\Big\{log\;\by^T[\bI_n-diag(\bA_\lambda)-\frac{1}{n}\bI_n]^{-1}(\bI_n-\bA_\lambda)[\bI_n-diag(\bA_\lambda)-\frac{1}{n}\bI_n]^{-1}\by\Big\}
\end{align*}


The value $K$ influences bias and variance of cross-validation. With $K=n$, the cross-validation estimator is approximately unbiased for the true (expected) prediction error because it almost use all the data in each training set. Therefore, it can have high variance because $n$ training sets are so similar to one another. Additionally, the computational burden is also considerable, requiring $n$ applications of the learning method. On the other hand, with larger $K$ such as 5 or 10, cross-validation will have lower variance, but making bias a problem.


\paragraph{AIC and small sample correction}\mbox{}\\
Based on the idea of "model fit + model complexity", Akaike's Information Criterion (AIC) choose $\lambda$ by minimizing,
\begin{align*}
AIC=&2(p+2)-2ln(\hat{L})\\
=&2(p+2)-2[-\frac{n}{2}log (\hat{\sigma}^2)-\frac{n}{2}log (2\pi)-\frac{1}{2\hat{\sigma}^2}\by^T(\bI_n-\bA_\lambda)\by]\\
=&2(p+2)+n log[\frac{1}{n}\by^T(\bI_n-\bA_\lambda)\by]+n+n log(2\pi)
\end{align*}
Drop the constant $n$ and divide it by $n$, we obtain our objective function:
\begin{align}
\lambda_{AIC}=\underset{\lambda \in \Lambda}{argmin}\Big\{log\; \by^T(\bI_n-\bA_\lambda)\by+\frac{2[tr(\bA_\lambda)+2)]}{n}\Big\}
\end{align}
When $n$ is small, extreme overfitting is possible, giving small bias/ large variance estimates. The small-sample correction of AIC is derived by minimizing minus 2 times expected log likelihood, where we plug in $\bA_\lambda$ and $\hat{\sigma}^2$. In this case, we obtain our small-sample size objective function AICc:
\begin{align}
\lambda_{AICc}=\underset{\lambda \in \Lambda}{argmin}\Big\{log\; \by^T(\bI_n-\bA_\lambda)\by+\frac{2[tr(\bA_\lambda)+2)]}{n-tr(\bA_\lambda)-3}\Big\}
\end{align}
Compare (2.1.5) and (2.1.6), it is easy to tell that AICc considers more of the model complexity since $\frac{n}{n-tr(\bA_\lambda)-3}>1$. It makes sense intuitively because it need to shrink more to prevent small bias/ large variance estimates.

\paragraph{GCV and small sample correction}\mbox{}\\
In (2.1.4), if we approximate each $A_{\lambda[ii]}$ with their mean $\frac{tr(\bA_\lambda)}{n}$, in a sense that we give equal weight to all observations. We get our GCV objective function:
\begin{align}
\lambda_{GCV}=\underset{\lambda \in \Lambda}{argmin}\Big\{log\; \by^T(\bI_n-\bA_\lambda)\by-2ln[1-\frac{tr(\bA_\lambda)}{n}-\frac{1}{n}]\Big\}
\end{align}
The "$-\frac{1}{n}$" terms in (2.1.7) is because GCV counts $\beta_0$ as part of model complexity, but not $\sigma^2$. This motivates the proposed small-sample correction to GCV, which does count $\sigma^2$ as a parameter:
\begin{align}
\lambda_{GCVc}=\underset{\lambda \in \Lambda}{argmin}\Big\{log\; \by^T(\bI_n-\bA_\lambda)\by-2ln[1-\frac{tr(\bA_\lambda)}{n}-\frac{2}{n}]_+\Big\}
\end{align}
Under this situation, perfect fit of the observations to the predictions, given by $\lambda=0$, cannot occur.

\paragraph{GMPML-based selection}\mbox{}\\
Suppose we relate $Y$ and $\bX$ by a linear model, $Y=\beta_0+\phi(\bX)^T\bbeta+\sigma \epsilon$, with $\epsilon \sim N(0, 1)$, where $\phi(\bx)$ is a function mapping a $D$-dimensional input vector $\bx$ into an $p$-dimensional feature space. If we assume $\bbeta$ are jointly and independently normal with mean zero and variance $\sigma^2/\lambda$, the penalty term matches the negative normal log-density, up to a normalizing constant not depending on $\bbeta$:
\begin{align*}
p_\lambda(\bbeta, \sigma^2)=\frac{\lambda}{2\sigma^2}\bbeta^T\bbeta-\frac{p}{2}log(\lambda)+\frac{p}{2}log(\sigma^2)
\end{align*}
One can consider a marginal likelihood, where $\lambda$ is interpreted as the variance component of a mixed-effects model:
\begin{align*}
m(\lambda, \sigma^2)=&log \int_{\bbeta} exp\{ l(\bbeta, \sigma^2)-p_\lambda(\bbeta, \sigma^2)\}d\bbeta \\=&-\frac{1}{2\sigma^2}\by^T(\bI_n-\bA_\lambda)\by-\frac{n}{2}log(\sigma^2)+\frac{1}{2}log \mid \bI_n-\bA_\lambda \mid
\end{align*}
From this, $\by |\lambda, \sigma^2$ is multivariate normal with mean $\mathbf{0}_n$ and covariance $\sigma^2(\bI_n-\bA_\lambda)^{-1}$. The maximum profile marginal likelihood (MPML) estimate profiles $m(\lambda, \sigma^2)$ over $\sigma^2$, replacing each instance with $\hat{\sigma}_\lambda^2=\by^T(\bI_n-\bA_\lambda)\by/n$, and maximized the "concentrated" log-likelihood, $m(\lambda, \hat{\sigma}_\lambda^2)$:
\begin{align*}
\lambda_{MPML}=\underset{\lambda \in \Lambda}{argmin}\Big\{log\; \by^T(\bI_n-\bA_\lambda)\by-\frac{1}{n}log \mid \bI_n-\bA_\lambda \mid \Big\}
\end{align*}
Generalized MPML adjusts the penalty to account for estimation of regression parameter $\beta_0$ that is not marginalized, resulting in one degree of freedom:
\begin{align}
\lambda_{GMPML}=\underset{\lambda \in \Lambda}{argmin}\Big\{log\; \by^T(\bI_n-\bA_\lambda)\by-\frac{1}{n-1}log \mid \bI_n-\bA_\lambda \mid \Big\}
\end{align}

\vspace{2em}
\paragraph{Discussion on GCV and GMPML, more...}\mbox{}\\
If we denote $\bU_K$ and $\{\eta_{K,j}\}_{j=1}^n$ the eigenvector and eigenvalues of $\bK$, then $\bI_n-\bA_\lambda$ adopts the form:
\begin{align*}
\bI_n-\bA_\lambda=\bU_K diag(\frac{\lambda}{\eta_{K,j}+\lambda})\bU_K^T
\end{align*}
As we can see from the two objective functions, $\lambda_{GCV}$ and $\lambda_{GMPML}$ share the same "model fit" term $log\; \by^T(\bI_n-\bA_\lambda)\by$. Their difference focuses on "model complexity" terms, which is maximizing $2log[1-\frac{tr(\bA_\lambda)}{n}-\frac{1}{n}]$ for $\lambda_{GCV}$ and $\frac{1}{n-1}log \mid \bI_n-\bA_\lambda \mid$ for $\lambda_{GMPML}$. Up to a constant not depending on $\lambda$, the difference boils down to,
\begin{align}
&GCV: \; log[tr(\bI_n-\bA_\lambda)]=log[\sum_{j=1}^n\frac{\lambda}{\eta_{K,j}+\lambda}]\\
&GMPML: \; log\mid \bI_n-\bA_\lambda\mid=log[\prod_{j=1}^n\frac{\lambda}{\eta_{K,j}+\lambda}]
\end{align}
Take derivative of these two equations with respect to $\eta_{K,j}$,
\begin{align*}
\frac{\partial }{\partial \eta_{K,j}}log[tr(\bI_n-\bA_\lambda)]&=-\frac{\lambda}{(\eta_{K,j}+\lambda)^2[\sum_{j=1}^n\frac{\lambda}{\eta_{K,j}+\lambda}]}\\
\frac{\partial }{\partial \eta_{K,j}}log\mid \bI_n-\bA_\lambda\mid&=-\frac{1}{\eta_{K,j}+\lambda}
\end{align*}

Take derivative of these two equations with respect to $\lambda$,
\begin{align*}
\frac{\partial }{\partial \lambda}log[tr(\bI_n-\bA_\lambda)]&=\frac{\sum_{j=1}^n\frac{\eta_{K,j}}{(\eta_{K,j}+\lambda)^2}}{\sum_{j=1}^n\frac{\lambda}{\eta_{K,j}+\lambda}}\\
\frac{\partial }{\partial \lambda}log\mid \bI_n-\bA_\lambda\mid&=\sum_{j=1}^n\frac{\eta_{K,j}}{\lambda(\eta_{K,j}+\lambda)}
\end{align*}


\subsubsection*{{1.3. ensemble strategy}}

\paragraph{cross-validation}\mbox{}\\
After obtaining the estimated errors $\{\hat{\epsilon}_d\}_{d=1}^D$, we estimate the ensemble weights $\bu=\{u_d\}_{d=1}^D$ such that it minimizes the overall error:
\begin{align*}
\hat{\bu}=\underset{\bu \in \Delta}{argmin}\parallel \sum_{d=1}^Du_d\hat{\epsilon}_d\parallel^2 \quad where\; \Delta=\{\bu | \bu \geq 0, \parallel \bu \parallel_1=1\}
\end{align*}
Then produce the final ensemble prediction:
\begin{align*}
\hat{\bh}=\sum_{d=1}^D \hat{u}_d \bh_d=\sum_{d=1}^D \hat{u}_d \bA_{d,\hat{\lambda}_d}\by=\hat{\bA}\by
\end{align*}
where $\hat{\bA}=\sum_{d=1}^D \hat{u}_d \bA_{d,\hat{\lambda}_d}$ is the ensemble matrix.

\paragraph{simple averaging}\mbox{}\\
Motivated by existing literature in omnibus kernel$^{2}$, we propose another way to obtain the ensemble matrix by simply choosing unsupervised weights $u_d=1/D$ for $d=1,2,...D$.

\subsubsection*{{1.4. kernel choice}}
\paragraph{types of kernel}\mbox{}\\
\paragraph{characterization as a function class}\mbox{}\\
\paragraph{spectral property}\mbox{}\\
\subsection*{{2. Hypothesis Test}}
\setcounter{equation}{0}
\renewcommand{\theequation}{2.2.\arabic{equation}}
\subsubsection*{{2.1. Asymptotic Test}}
We use the classical variance component test to construct a testing procedure for the hypothesis about Gaussian process function:
\begin{align}
H_0: h \in \Hsc_0.
\end{align}
We first translate above hypothesis into a hypothesis in terms of model parameters. The key of our approach is to assume that $h$ lies in a RKHS generated by a \textsl{garrote kernel function}$k_{\delta}(\bz, \bz')$, which is constructed by including an extra \textsl{garrote parameter} $\delta$ to a given kernel function. When $\delta=0$, the garrote kernel function $k_0(\bx, \bx')=k_{\delta}(\bx, \bx')\mid _{\delta=0}$ generates exactly $\Hsc_0$, the space of functions under the null hypothesis. In order to adapt this general hypothesis to their hypothesis of interest, practitioners need only to specify the form of the garrote kernel so that $\Hsc_0$ corresponds to the null hypothesis. For example, if $k_{\delta}(\bx)=k(\delta * x_1, x_2,..., x_p), \delta=0$ corresponds to the null hypothesis $H_0: h(\bx)=h(x_2,...,x_p)$, i.e. the function $h(\bx)$ does not depend on $x_1$. As a result, the general hypothesis is equivalent to:
\begin{align}
H_0: \delta=0.
\end{align}
We now construct a test statistic $\hat{T}_0$ for (2.2.2) by noticing that the garrote parameter $\delta$ can be treated as a variance component parameter in the linear mixed model. This is because the Gaussian process under garrote kernel can be formulated into below LMM:
\begin{align*}
\by=\bmu+\bh+\bepsilon \quad where \quad \bh \sim N(\mathbf{0}, \tau \bK_\delta) \quad \bepsilon \sim N(\mathbf{0}, \sigma^2\bI)
\end{align*}
where $\bK_\delta$ is the kernel matrix generated by $k_{\delta}(\bz, \bz')$. Consequently, we can derive a variance component test for $H_0$ by calculating the square derivative of $L_{REML}$ with respect to $\delta$ under $H_0$:
\begin{align}
\hat{T}_0=\hat{\tau}*(\by-\hat{\bmu})^T\bV_0^{-1}[\partial \bK_0]\bV_0^{-1}(\by-\hat{\bmu})
\end{align}
where $\bV_0=\hat{\sigma}^2\bI+\hat{\tau}\bK_0$. In this expression, $\bK_0=\bK_\delta \mid_{\delta=0}$, and $\partial \bK_0$ is the null derivative kernel matrix whose $(i, j)^{th}$ entry is $\frac{\partial }{\partial \delta}k_\delta(\bx, \bx') \mid_{\delta=0}$.
As discussed previously, misspecifying the null kernel function $k_0$ negatively impacts the performance of the resulting hypothesis test. To better understand the mechanism at play, we express the test statistic $\hat{T}_0$ from (2.2.3) in terms of the model residual $\hat{\bepsilon}=\by-\hat{\bmu}-\hat{\bh}$:
\begin{align}
\hat{T}_0=(\frac{\hat{\tau}}{\hat{\sigma}^4})*\hat{\bepsilon}^T[\partial \bK_0]\hat{\bepsilon},
\end{align}
where we have used the fact $\bV_0^{-1}(\by-\hat{\bmu})=(\hat{\sigma}^2)^{-1}(\hat{\bepsilon})$. As shown, the test statistic $\hat{T}_0$ is a scaled quadratic-form statistic that is a function of the model residual. If $k_0$ is too restrictive, model estimates will \textbf{underfit} the data even under the null hypothesis, introducing extraneous correlation among the $\hat{\epsilon}_i$'s, therefore leading to overestimated $\hat{T}_0$ and eventually underestimated p-valueunder under the null. In this case, the test procedure will frequently reject the null hypothesis (i.e. suggest the existence of nonlinear interaction) even when there is in fact no interaction, yielding an invalid test due to \textbf{inflated Type I error}. On the other hand, if $k_0$ is too flexible, model estimates will likely \textbf{overfit} the data in small samples, producing underestimated residuals, an underestimated test statistic, and overestimated p-values. In this case, the test procedure will too frequently fail to reject the null hypothesis (i.e. suggesting there is no interaction) when there is in fact interaction, yielding an insensitive test with \textbf{diminished power}.\\
The null distribution of $\hat{T}$ can be approximated using a scaled chi-square distribution $\kappa \chi_\nu^2$ using Satterthwaite method by matching the first two moments of $T$:
\begin{align*}
\kappa * \nu=E(T)=\hat{\tau} * tr(\bV_0\partial \bK_0)
 \quad 2*\kappa^2*\nu=Var(T)=\hat{\bI}_{\delta\delta}
\end{align*}
with solution:
\begin{align*}
\hat{\kappa}=\hat{\bI}_{\delta\delta}/[\hat{\tau}*tr(\bV_0^{-1}\partial \bK_0)] \quad \hat{\nu}=[\hat{\tau}*tr(\bV_0^{-1}\partial \bK_0)]^2/(2*\hat{\bI}_{\delta\theta})
\end{align*}
where $\hat{\bI}_{\delta\delta}=\bI_{\delta\delta}-\bI_{\delta\theta}^T\bI_{\theta\theta}^{-1}\bI_{\delta\theta}$ is the efficient information of $\delta$ under REML. $\bI_{\delta\delta}$, $\bI_{\theta\theta}$ and $\bI_{\delta\theta}$ are submatrices of the REML information matrix. 
Numerically more accurate, but computationally less efficient approximation methods are also available.\\
Finally, the p-value of this test is calculated by examining the tail probability of $\hat{\kappa} \chi_{\hat{\nu}}^2$:
\begin{align*}
p=P(\hat{\kappa} \chi_{\hat{\nu}}^2>\hat{T})=P(\chi_{\hat{\nu}}^2>\hat{T}/\hat{\kappa})
\end{align*}
A complete summary of the proposed testing procedure is available in Algorithm 2.\\
In light of the discussion about model misspecification in Introduction section, we highlight the fact that our proposed test (2.2.3) is robust against model misspecification under the alternative, since the calculation of test statistics do not require detailed parametric assumption about $k_\delta$. However, the test is NOT robust against model misspecification under the null, since the expression of both test statistic $\hat{T}_0$ and the null distribution parameters $(\hat{\kappa}, \hat{\nu})$ still involve the kernel matrices generated by $k_0$ (see Algorithm 2). We address this problem by proposing a robust estimation procedure for the kernel matrices under the null.

\begin{algorithm}
\caption{Variance Component Test for $h \in \Hsc_0$} 
\label{alg:cvek}
\begin{algorithmic}[1]
\Procedure{VCT FOR INTERACTION}{}\newline
\textbf{Input:} Null Kernel Matrix $\bK_0$, Derivative Kernel Matrix $\partial \bK_0$, Data $(y, \bx)$\newline
\textbf{Output:} Hypothesis Test p-value $p$\newline
$\#$ \texttt{Stage 1: Estimate Null Model using REML}
\State{$(\hat{\bmu}, \hat{\tau}, \hat{\sigma}^2)=argmin L_{REML}(\bmu, \tau, \sigma^2|\bK_0)$}
\newline
$\#$ \texttt{Stage 2: Compute Test Statistic and Null Distribution Parameters}
\State{$\hat{T}_0=\hat{\tau}*(\by-\bX\hat{\bbeta})^T\bV_0^{-1}\partial \bK_0\bV_0^{-1}(\by-\bX\hat{\bbeta})$}
\State{$\hat{\kappa}=\hat{\bI}_{\delta\delta}/[\hat{\tau}*tr(\bV_0^{-1}\partial \bK_0)]$, \quad $\hat{\nu}=[\hat{\tau}*tr(\bV_0^{-1}\partial \bK_0)]^2/(2*\hat{\bI}_{\delta\theta})$}
\newline
$\#$ \texttt{Stage 3: Compute p-value and reach conclusion}
\State{$p=P(\hat{\kappa} \chi_{\hat{\nu}}^2>\hat{T})=P(\chi_{\hat{\nu}}^2>\hat{T}/\hat{\kappa})$}
\EndProcedure
\end{algorithmic}
\end{algorithm}

\subsubsection*{{2.2. Bootstrap Test}}
In practice, the sample size of the collected data is always small. To make valid inferences about a population from the sample, we need to perform resampling. Commonly used methods in small sample size are permutation tests and bootstrap.\\
\paragraph{Permutation Tests}\mbox{}\\
Permutation tests are very popular in genomic research. They are simple to compute where analytic approaches may be intractable, and can be exact where analytic results may be only approximate. However, the main limitation of permutation tests is that they are only applicable when the null hypothesis being tested specifies a suitable group of permutations under which the distribution of the data would be unaffected. To illustrate this idea, consider a test for interaction between the effects of the genetic polymorphism $G$ and a environmental exposure $E$ on an outcome $Y$. The null hypothesis is that the interaction term is zero.
\begin{align}
E[Y]=\beta_0+\beta_GG+\beta_EE
\end{align}
An alternative hypothesis of interest may be that,
\begin{align*}
E[Y]=\beta_0+\beta_GG+\beta_EE+\delta G*E
\end{align*}
For a valid permutation test of the hypothesis of no interaction, we would require a group of permutations that exactly preserves both $\beta_G$ and $\beta_E$ in (2.2.5), but also ensures $\delta=0$. In general it is impossible to construct such a group of permutations, as demonstrated by Edgington (1987, Chap. 6). If the permutations fix $G$ and $E$ they will not give $\delta=0$, and if they do not fix $G$ and $E$ they will not preserve the relationship between $G$ and $E$ and so will not preserve $\beta_G$ and $\beta_E$.

\paragraph{Bootstrap Test}\mbox{}\\
Instead we can use parametric bootstrap, which can give valid tests with moderate sample sizes and requires similar computational effort to a permutation test.\\
Testing in a regression model framework requires computing the distribution of the test statistic under sampling from the null-hypothesis model. A good approximation to the distribution of the test statistic under sampling from the true-hypothesis model is the distribution of the test statistic under sampling from the fitted null-hypothesis model. For instance, when testing (2.2.2), we first fit the model under the null,
\begin{align}
E(\by^\star)=\bK_0(\bK_0+\lambda \bI)^{-1}\by=\bA_0\by
\end{align}
and generate $\bY^\star$ for each individuals with a random noise, whose variance is also estimated. We then compute the test statistic for this simulated sample, and repeat this process $B$ times. The empirical distribution these provide is an estimate of the test statistic's distribution under the null. Correspondingly, p-values are calculated as the proportion of simulated test statistics that are most extreme than the observed value.\\
If the distribution of the test statistic depends smoothly on the regression parameter values, which is true in all standard examples, this ‘parametric bootstrap’ approach gives an asymptotically valid test(Davison \& Hinkley 1997, 4.2.3). Like the classical bootstrap, it samples from a distribution based on the observed data, but the simulations are from a fitted parametric model rather than the empirical distribution. To obtain a valid test, the fitted parametric model is chosen so that the null hypothesis is satisfied.
A complete summary of the proposed testing procedure is available in Algorithm 3.\\

\begin{algorithm}
\caption{Parametric Bootstrap Test} 
\label{alg:cvek}
\begin{algorithmic}[1]
\Procedure{Parametric Bootstrap Test}{}\newline
\textbf{Input:} Null Kernel Matrix $\bK_0$, Derivative Kernel Matrix $\partial \bK_0$, Data $(\by, \bx)$\newline
\textbf{Output:} Hypothesis Test p-value $p$\newline
$\#$ \texttt{Stage 1: Estimate Null Model using Gaussian Process Regression}
\State{$\hat{\bmu}=\bA_0\by, \quad \hat{\sigma}^2=\frac{\by^T(\bI_n-\bA_0)\by}{n-tr(\bA_0)}$}
\newline
$\#$ \texttt{Stage 2: Sample response from the fitted model obtain in Step 1}
\newline
$\#$ \texttt{and compute the test statistic based on fitting the alternative}
\newline
$\#$ \texttt{model, repeat for $B$ times}
\For{$b= 1$ to $B$}
\State{$\by^\star=\hat{\bmu}+\bepsilon,\quad \bepsilon \sim N(0, \hat{\sigma}^2)$}
\State{$\hat{T}_{0b}=\hat{\tau}*(\by^\star-\hat{\bmu})^T\bV_0^{-1}\partial \bK_0\bV_0^{-1}(\by^\star-\hat{\bmu})$} 
\EndFor 
\newline
$\#$ \texttt{Stage 3: Compute the test statistic for the original data, based}
\newline
$\#$ \texttt{on fitting the alternative hypothesis model}
\State{$\hat{T}_0=\hat{\tau}*(\by-\hat{\bmu})^T\bV_0^{-1}\partial \bK_0\bV_0^{-1}(\by-\hat{\bmu})$}
\newline
$\#$ \texttt{Stage 4: Compute p-value and reach conclusion}
\State{$p=\frac{1}{B}\sum_{b=1}^BI(\hat{T}_{0b}>\hat{T}_0)$}
\EndProcedure
\end{algorithmic}
\end{algorithm}


\section*{{\bf 3. Simulation}}
\section*{{\bf 4. Conclusion}}















%----------------------------------------------------------------------------------------
\end{document}